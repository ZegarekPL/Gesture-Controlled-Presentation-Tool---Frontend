\documentclass[12pt,a4paper]{article}
\usepackage[T1]{fontenc}
\usepackage[utf8]{inputenc}
\usepackage[polish]{babel}
\usepackage{graphicx}
\usepackage{listings}
\usepackage{xcolor}
\usepackage{hyperref}
\usepackage{geometry}
\usepackage{enumitem}

\geometry{
    a4paper,
    total={170mm,257mm},
    left=20mm,
    top=20mm,
}

\definecolor{codegreen}{rgb}{0,0.6,0}
\definecolor{codegray}{rgb}{0.5,0.5,0.5}
\definecolor{codepurple}{rgb}{0.58,0,0.82}
\definecolor{backcolour}{rgb}{0.95,0.95,0.92}

\lstdefinestyle{mystyle}{
    backgroundcolor=\color{backcolour},   
    commentstyle=\color{codegreen},
    keywordstyle=\color{magenta},
    numberstyle=\tiny\color{codegray},
    stringstyle=\color{codepurple},
    basicstyle=\tiny\ttfamily,
    breakatwhitespace=false,
    breaklines=true,
    captionpos=b,
    keepspaces=true,
    numbers=left,
    numbersep=5pt,
    showspaces=false,
    showstringspaces=false,
    showtabs=false,
    tabsize=2
}

\lstset{style=mystyle}

\lstdefinelanguage{JavaScript}{
  keywords={async, const, let, var, function, return, if, else, for, while, do, switch, case, break, continue, new, try, catch, throw, typeof, instanceof},
  keywordstyle=\color{magenta}\bfseries,
  ndkeywords={class, export, boolean, throw, implements, import, this, await},
  ndkeywordstyle=\color{blue}\bfseries,
  identifierstyle=\color{black},
  sensitive=false,
  comment=[l]{//},
  morecomment=[s]{/*}{*/},
  commentstyle=\color{codegreen}\ttfamily,
  stringstyle=\color{codepurple}\ttfamily,
  morestring=[b]',
  morestring=[b]",
  basicstyle=\tiny\ttfamily
}

\begin{document}
\begin{titlepage}
    \begin{center}
        \vspace*{2cm}
        {\huge\bfseries Prezentacja sterowana gestami\par}
        \vspace{1.5cm}
        {\Large\itshape Dokumentacja Projektowa\par}
        \vspace{1.5cm}
        {\large Autor:\par}
        {\large\bfseries Filip Kula, Wiktor Mazur\par}
        \vspace{1cm}
        {\large Przedmiot:\par}
        {\large\itshape Interakcja Człowiek - Komputer\par}
    \end{center}
\end{titlepage}

\newpage

\section{Wprowadzenie}
Projekt zakłada wykonanie prostego systemu do prezentacji zarządzanego gestami dłoni.

\section{Architektura systemu}
\subsection{Komponenty główne}
\begin{itemize}
    \item \textbf{Frontend}: Next.js + React
    \item \textbf{Detekcja gestów}: MediaPipe Hands
    \item \textbf{Routing}: Next.js Router
    \item \textbf{Zarządzanie stanem}: React Hooks
\end{itemize}

\subsection{Struktura projektu}
\begin{itemize}
    \item docs/
    \item src/
    \begin{itemize}
        \item app/
        \begin{itemize}
            \item layout.tsx
            \item page.tsx
        \end{itemize}
        \item components/
        \begin{itemize}
            \item HandGestureControls.tsx
            \item Sidebar.tsx
            \item SlideContent.tsx
        \end{itemize}
    \end{itemize}
\end{itemize}

\section{Funkcjonalności}
\subsection{Kontrola gestów}
System wykorzystuje MediaPipe Hands do detekcji i śledzenia ruchów dłoni. Zaimplementowane gesty:
\begin{itemize}
    \item Wskazanie w lewo - wcześniejszy slajd
    \item Wskazanie w prawo - następny slajd
\end{itemize}

\section{Implementacja}
\subsection{Detekcja gestów}
\begin{lstlisting}[language=JavaScript]
const gestureDetection = async (frame) => {
  const hands = new mp.Hands({
    maxNumHands: 1,
    modelComplexity: 1,
    minDetectionConfidence: 0.5,
    minTrackingConfidence: 0.5
  });
  
  // Przetwarzanie klatki wideo
  const results = await hands.process(frame);
  return results.multiHandLandmarks;
};
\end{lstlisting}

\subsection{Komponent rysowania}
\begin{lstlisting}[language=JavaScript]
const DrawingCanvas = ({ slideId }) => {
  const canvasRef = useRef(null);
  const [isDrawing, setIsDrawing] = useState(false);
  
  // Logika rysowania
  const startDrawing = (e) => {
    setIsDrawing(true);
    const ctx = canvasRef.current.getContext('2d');
    ctx.beginPath();
    ctx.moveTo(e.clientX, e.clientY);
  };
};
\end{lstlisting}

\subsection{Wymagania programowe}
\begin{itemize}
    \item Nowoczesna przeglądarka internetowa (Chrome 88+, Firefox 85+, Safari 14+)
    \item Obsługa WebGL
    \item Włączona obsługa JavaScript
    \item Dostęp do kamery
\end{itemize}

\section{Instalacja i uruchomienie}

\subsection{Za pomocą Docker}

1. Pobierz najnowszą wersję obrazu Dockera z repozytorium na GitHubie:
\url{https://github.com/ZegarekPL/Gesture-Controlled-Presentation-Tool---Frontend/releases}

2. Po pobraniu pliku z obrazem, załaduj go do Dockera, używając poniższej komendy:

\begin{lstlisting}[language=bash]
    docker load -i gesture_controlled_presentation_tool_frontend-v0.1.1.tar
\end{lstlisting}

3. Następnie uruchom aplikację w kontenerze, używając tej komendy:

\begin{lstlisting}[language=bash]
    docker run -d -p 3000:3000 gesture_controlled_presentation_tool_frontend:v0.1.1
\end{lstlisting}

4. Po uruchomieniu kontenera, aplikacja będzie dostępna pod adresem: \url{http://localhost:3000}.

\subsection{Za pomocą VSCode/WebStorm}
\subsubsection{Instalacja zależności}
\begin{lstlisting}[language=bash]
npm install
# lub
yarn install
\end{lstlisting}

\subsubsection{Uruchomienie}
\begin{lstlisting}[language=bash]
npm run dev
# lub
yarn dev
\end{lstlisting}

\end{document}